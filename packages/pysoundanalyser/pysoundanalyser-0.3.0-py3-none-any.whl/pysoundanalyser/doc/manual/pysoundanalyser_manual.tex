\documentclass[a4paper,12pt]{report}
\usepackage[pdftex]{graphicx}

%\usepackage[hmargin={3.5cm}]{geometry}
\usepackage{mathpazo,bm}
%\usepackage[T1]{fontenc}
%\usepackage{lmodern} %lmodern,euler
\usepackage{afterpage}
\usepackage{booktabs}
\usepackage{longtable}
\usepackage{multicol}
\usepackage{fancyvrb}
%\usepackage{graphicx}
\usepackage{caption}
%\usepackage{keystroke}
\usepackage{makeidx}
%\usepackage{floatflt}
\usepackage{wrapfig}
\captionsetup[table]{aboveskip=5pt,belowskip=5pt,position=top,margin=10pt,font=small,labelfont=bf,format=hang}
\captionsetup[figure]{aboveskip=0pt,belowskip=0pt,position=top,margin=10pt,font=small,labelfont=bf,format=hang}

\makeindex

%\setlength{\parindent}{10pt}
\setcounter{secnumdepth}{2}

\usepackage[hyperindex, bookmarks=true, bookmarksnumbered=true, breaklinks=true]{hyperref}
\hypersetup{pdftitle=pysoundanalyser manual,
            pdfauthor=Samuele Carcagno,
            pdfsubject=Manual for pysoundanalyser,
            pdfpagemode=UseOutlines,
            pdfstartview=FitH,
            pdfkeywords={pysoundanalyser, python, sound, analyser},
            colorlinks=true,
            linkcolor=black,
            citecolor=black,
            filecolor=black,
            urlcolor=blue  
} 
%pdfpagemode=none #for no bookmarks


\begin{document}
\pagenumbering{gobble}
\begin{center}
\Huge \textsc{pychoacoustics manual} \\
\rule{\textwidth}{1mm}
\end{center}
{\flushright \normalsize Version 0.2}\\
\begin{center}
\vspace{.2cm}

\Large Samuele Carcagno
\normalsize \\
\vspace{.3cm}


\href{mailto:sam.carcagno@gmail.com}{\textcolor{black}{sam.carcagno@gmail.com}}
\end{center}

%\vspace{15cm}
%\noindent \footnotesize{samuele\_carcagno@yahoo.co.uk}
%\href{mailto:samuele_carcagno@yahoo.co.uk}{samuele\_carcagno@yahoo.co.uk}

\clearpage
%\flushleft
\noindent Copyright \copyright  2012--2013  Samuele Carcagno.\\
      Permission is granted to copy, distribute and/or modify this document
      under the terms of the GNU Free Documentation License, Version 1.2
      or any later version published by the Free Software Foundation;
      with no Invariant Sections, no Front-Cover Texts, and no Back-Cover
	 Texts.  A copy of the license is included in the section entitled ``GNU      Free Documentation License''.

\vspace{2cm}
\noindent \textbf{Disclaimer}: This document comes with NO WARRANTY whatsoever of being correct in any of its parts.
This document is work in progress.
\pagenumbering{gobble}
%%% Local Variables: 
%%% mode: latex
%%% TeX-master: "main"
%%% End: 

\reversemarginpar
\newcommand{\sq}{\texttt{"}}           %{\textquotedbl}
%\newcommand{\pycho}{\texttt{pychoacoustics}}
\renewcommand{\bibname}{References}

\newcommand*{\main}[1]{\textbf{\hyperpage{#1}}}

%\pagenumbering{gobble}
\begin{center}
\Huge \textsc{pychoacoustics manual} \\
\rule{\textwidth}{1mm}
\end{center}
{\flushright \normalsize Version 0.2}\\
\begin{center}
\vspace{.2cm}

\Large Samuele Carcagno
\normalsize \\
\vspace{.3cm}


\href{mailto:sam.carcagno@gmail.com}{\textcolor{black}{sam.carcagno@gmail.com}}
\end{center}

%\vspace{15cm}
%\noindent \footnotesize{samuele\_carcagno@yahoo.co.uk}
%\href{mailto:samuele_carcagno@yahoo.co.uk}{samuele\_carcagno@yahoo.co.uk}

\clearpage
%\flushleft
\noindent Copyright \copyright  2012--2013  Samuele Carcagno.\\
      Permission is granted to copy, distribute and/or modify this document
      under the terms of the GNU Free Documentation License, Version 1.2
      or any later version published by the Free Software Foundation;
      with no Invariant Sections, no Front-Cover Texts, and no Back-Cover
	 Texts.  A copy of the license is included in the section entitled ``GNU      Free Documentation License''.

\vspace{2cm}
\noindent \textbf{Disclaimer}: This document comes with NO WARRANTY whatsoever of being correct in any of its parts.
This document is work in progress.
\pagenumbering{gobble}
%%% Local Variables: 
%%% mode: latex
%%% TeX-master: "main"
%%% End: 

\pagenumbering{gobble}


\pagenumbering{roman}
\tableofcontents

\clearpage
\pagenumbering{arabic}

\chapter{What is \texttt{pychoacoustics}?}

\texttt{pychoacoustics} is a software for programming and running experiments in auditory psychophysics (psychoacoustics). The software contains a set of predefined experiments that can be immediately run after installation. Importantly \texttt{pychoacoustics} is designed to be extensible so that users can add new custom experiments with relative ease. Custom experiments are written in Python, a programming language renowned for its clarity and ease of use. The application is divided in two graphical windows a) the ``response box'', shown in Figure~\ref{fig:response_box}, with which listeners interact during the experiment b) the control window, shown in Figure~\ref{fig:control_window}, that contains a series of widgets (choosers, text field and buttons) that are used by the experimenter to set all of the relevant experimental parameters, which can also be stored and later reloaded into the application. %Writing experiments for \texttt{pychoacoustics} gives immediate access to an extensive set of facilities for the selection, storage and retrieval of experimental parameters, stimulus presentation, randomization of experimental blocks, as well as storage and processing of experimental responses.

\begin{figure}[!h]
   \caption{The Response Box}
   \centering
   \includegraphics[scale=0.6]{Figures/response_box.png}
   \label{fig:response_box}
 \end{figure}

\begin{figure}[!h]
   \caption{The Control Window}
   \centering
   \includegraphics[scale=0.55]{Figures/control_window.png}
   \label{fig:control_window}
 \end{figure}

I started writing \texttt{pychoacoustics} for fun and for the sake of learning around 2008 while doing my PhD with Professor Chris Plack at Lancaster University. At that time we were using in the lab a MATLAB program called the ``Earlab'' written by Professor Plack. \texttt{pychoacoustics} has been greatly influenced and inspired by the ``Earlab'', and it reproposes many of the same features that the ``Earlab'' provides. For this reason, as well as for the patience he had to teach me audio programming I am greatly indebted to Professor Plack. %The ``Earlab'' has a venerable history, and was ported to various platforms and programming languages before reaching its MATLAB incarnation. \texttt{pychoacoustics} was initially written using the wxWidgets toolkit, and was 



%%% Local Variables: 
%%% mode: latex
%%% TeX-master: "pychoacoustics_manual"
%%% End: 


\chapter{User Interface}

\subsubsection{Load Sound}
The ``Load Sound'' button allows you to load a wav file into the program. Currently only 16 and 32 bit wav files with one or two channels are supported. Note that the entire wav file is loaded in memory, this is fine and fast for wav files of short durations (tens of seconds), but longer sound files are going to consume huge amounts of RAM and may even halt your computer. If you need to work on long sound files, please use other software, like audacity.  

\subsubsection{Save As}
The ``Save as'' button allows you to save PSA sound objects as wav files. Currently it is only possible to save sounds as 16 or 32 bit wav files with one or two channels. If you choose a mono (1-channel) format, and multiple PSA sound objects have been selected for saving, they will be summed together before saving. If you choose a stereo (2-channels) format, ``right'' and ``left'' PSA sound objects will be saved to their respective channels in the wav file; if multiple ``right'' or ``left'' PSA sound objects have been selected, they will be summed before saving.

\subsubsection{Cut}
The ``Cut'' button allows you to remove segments of a sound waveform. The starting and ending points of the segments to be cut off can be specified in seconds, milliseconds, or sample numbers. When working with sample numbers, note that \texttt{PSA} indexing works exactly like numpy indexing. Examples:
\begin{verbatim}

>>> import numpy as np #import numpy
>>> sig = np.arange(10) #generate a 10-elements array
>>> x[0:5] #Select the first five samples
           #indexing starts from 0
array([0, 1, 2, 3, 4])
>>> sig[7:10] #select last 3 samples
array([7, 8, 9])
>>> sig[1:3] #select the second and third sample
array([1, 2])
>> sig[1:2] #select the second sample
array([1])
>>> sig[1:1] #note that if the start and end point
             #are the same nothing is selected
array([], dtype=int64)



\end{verbatim}

%%% Local Variables: 
%%% mode: latex
%%% TeX-master: "pysoundanalyser_manual"
%%% End: 



% \appendix
% \renewcommand{\arraystretch}{1.5}
% \include{}


% \cleardoublepage
% \phantomsection
% \addcontentsline{toc}{chapter}{References}
% \bibliographystyle{abbrv}
% \bibliography{r}
% \nocite{*}

% \cleardoublepage
% \phantomsection
% \addcontentsline{toc}{chapter}{Index}
% \printindex

\end{document}