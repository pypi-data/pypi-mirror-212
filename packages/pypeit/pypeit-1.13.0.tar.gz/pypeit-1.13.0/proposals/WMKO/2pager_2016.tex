\documentclass[11pt,preprint]{aastex}

\usepackage{latexsym}
\usepackage{fancybox}
\usepackage{graphicx}
\usepackage{amssymb}
\usepackage{color}
\usepackage{ulem}
\usepackage{float}
\usepackage{multicol}
\usepackage{enumitem}
%\usepackage{setspace}
\usepackage[parfill]{parskip}

\pretolerance=10000
\textwidth=6.4in
\textheight=8.9in
\voffset = 0.0in
\topmargin=0.0in
\headheight=0.00in
\hoffset = 0.0in
\headsep=0.20in
\oddsidemargin=0in
\evensidemargin=0in
\parindent=0em
\parskip=0.25ex
 
%\input{defs_xavier}
%\input{/u/xavier/bin/latex}
%\input{/u/xavier/bin/defs}

\newcommand{\npair}{50}
\def\ssp{\def\baselinestretch{1.0}\large\normalsize}
\newcommand{\oneskip}{\vskip \baselineskip}
\newcommand{\annrev}{ARA\&A}
\newcommand{\za}{z$_{ abs}$}
\newcommand{\zg}{z$_{ galaxy}$}
\newcommand{\zl}{$z_{emitter}$}
\newcommand{\zf}{z$_{df}$}
\newcommand{\avxi}{$\langle\xi\rangle$}
\newcommand{\omg}{$\Omega_{g}(z)$}
\newcommand{\tu}{$\tau( > M)$}
\newcommand{\mc}{$10^{12}M_{\odot}$}
\def \lnhi {$\log N_{HI}$}
\def \cmm  {cm$^{-2}$}
\def \cmmm {cm$^{-3}$}
\def \hfreq {$f_{\rm{HI}} (N,X)$}
\def \gamrate {$\Gamma_{912}$}
\def \fn {$f_{\rm{HI}} (N,X)$}
\def \lyaf {Ly--$\alpha$ forest}
\def \nqso  {seven}
\def \cmm  {\rm cm^{-2}}
\newcommand{\lya}{Ly$\alpha$}

\usepackage{multicol}

\newenvironment{my_itemize}{
\begin{itemize}
  \setlength{\itemsep}{1pt}
  \setlength{\parskip}{0pt}
  \setlength{\parsep}{0pt}}{\end{itemize}
}

\newenvironment{my_enumerate}{
\begin{enumerate}
  \setlength{\itemsep}{1pt}
  \setlength{\parskip}{0pt}
  \setlength{\parsep}{0pt}}{\end{enumerate}
}


% Attached is a 2-pager describing a data reduction pipeline (DRP)
% package -- PYPIT -- that we are developing for our scientific
% interests.  It has reached sufficient maturity that I write to
% explore WMKO/UCO funding to enhance its usage/impact on our 
% community.  Please review and hopefully discuss while the two
% of you are in Scotland.  Of course, we would be willing to make
% formal presentations to the SSC and UCOAC, although we consider
% this matter to be somewhat time critical.

	\pagestyle{myheadings}    % Go for customized headings
%\markright{Prochaska--U079 2008A Proposal---Quasars Probing Quasars}
	\markboth{\hfill PYPIT for WMKO/UCO (v1.0) \hfill}{\hfill
          PYPIT for WMKO/UCO (v1.0) \hfill}

\begin{document}

%%%%%%%%%%%%%%%%%%%%%%%%%%%%%%%%%%%%%%%%%%%%%%%%%%%%%%%%%%%%%
\vskip 0.2in

{}
\noindent
\uline{Brief Introduction:} \\
PYPIT is a public, Python-based data reduction pipeline
(DRP) package for astronomical optical spectroscopy.
Its lead developers are Ryan Cooke (U. Durham) and 
J. Xavier Prochaska (UC Observatories and UC Santa Cruz).
Its governing philosophy is to apply modern data reduction
techniques to a set of raw data files to produce
science-grade, calibrated spectra (1D spectra; 2D spectral
images) with minimal user interaction\footnote{The code
does generate quality assurance (QA) output to assess
performance.}.  Our efforts with PYPIT are primarily driven by 
our scientific interests and needs, but these
are broad and involve nearly every optical/IR spectrometer
on large aperture telescopes.

\parindent=2em

%%%%%%%%%%%%%%%%%%%%%%%%%%%%%%%%
\vskip 0.2in

\noindent 
\uline{An Opportunity for WMKO/UCO:} \\
For the past several years, the Keck Science Steering
Committee (SSC) and UCO Advisory Committee (UCOAC)
have expressed great interest in advancing the number
and quality of DRPs.  Indeed, it is a specific goal
within the soon-to-be-released Keck Strategic Plan.
Furthermore, both committees have identified enhanced
archives as priorities for WMKO and UCO.
JXP has been involved in these efforts since their 
inception and remains committed to the public distribution
of DRPs and archival data.

Over the past few months, PYPIT has developed into a 
DRP package that we are confident could greatly 
benefit the WMKO and UCO communities.  To an extent, 
this will continue to occur without WMKO or UCO support.
But we believe that now is the time when WMKO/UCO support
would have greatest impact on the future products.
And, given the typical delay between securing funding and
actual hiring,  the timing may even be somewhat behind.
Below, we describe a set of deliverables that 1~FTE could
accomplish.  We have especially emphasized ones that our 
team is less likely to focus on themselves.

Regarding JXP, he is offering to contribute a portion of his
UCO staff time to manage this FTE\footnote{It is very possible
that having two part-time staff members would be advantageous
to a single FTE.}.  For co-leader Cooke, we may need to
be more creative to compensate his time.
We emphasize that DRP development is a difficult and
rarely rewarding task for an academic at any level.
JXP will be especially mindful to protect the junior 
team members engaged in PYPIT.


Below is a list of task-specific development activities.
We would expect WMKO/UCO to provide input on priorities,
additional specific needs, etc.  At the end of this
document is a straw-man time-line for the release
order for specific instruments and modes.  We note
that the ones listed in Year~0 are already in development,
funded in part by UCO (DEIMOS, APF).  


\begin{my_enumerate}
\item Interface PYPIT to KOA
  \begin{my_enumerate}
  \item Interface Level 0 KOA products (naming, calibs) with PYPIT
  \item Deliver Level 1 products in KOA-ready format
  \item On-line views of the outputs, meta files
  \end{my_enumerate}
\item Interface PYPIT to Lick archive
\item ETC/throughput monitoring
  \begin{my_enumerate}
  \item With web-interfaces
  \end{my_enumerate}
\item Documentation, documentation, documentation
\item Calibraton archiving
  \begin{my_enumerate}
  \item Bias frames
  \item Flat field images
  \item Sky spectra
  \end{my_enumerate}
\item Source deblending
\item Testing 
  \begin{my_enumerate}
  \item Additional unit tests ($>90\%$ coverage)
  \item Additional test suites
  \end{my_enumerate}
\item Web-based QA
\item Quick reduce software at the telescope
  \begin{my_enumerate}
  \item S/N and spectrum plots `immediately' after a science exposure
  \item Without user knowledge of PYPIT!
  \end{my_enumerate}
\item Manage code architecture
\end{my_enumerate}


%%%%%%%%%%%%%%%%%%%%%%%%%%
\vskip 0.2in

\begin{center}
{\bf Timeline}
\end{center}

\begin{my_enumerate}
\item Instruments and modes for Year 0
  \begin{my_itemize}
  \item Keck/LRIS (both cameras)
    \begin{my_itemize}
    \item Longslit
    \item Multi-slit :: may require slitmask files for KOA ingestion
    \end{my_itemize}
  \item Keck/DEIMOS
    \begin{my_itemize}
    \item Longslit
    \item Multi-slit 
    \end{my_itemize}
  \item Lick/Kast
  \item Lick/APF
    \begin{my_itemize}
    \item Without iodine cell analysis
    \end{my_itemize}
  \end{my_itemize}
\item Year 1
  \begin{my_itemize}
  \item Keck/ESI
  \item Keck/HIRES
    \begin{my_itemize}
    \item Without iodine cell analysis
    \end{my_itemize}
  \item Keck/NIRSPEC
    \begin{my_itemize}
    \item Low-dispersion
    \end{my_itemize}
  \item Lick/Hamilton Spectrograph
  \end{my_itemize}
\item Year 2
  \begin{my_itemize}
  \item Keck/NIRES -- if delivered
  \item Keck/MOSFIRE -- if desired
  \item Keck/KCWI -- if desired
  \end{my_itemize}
\end{my_enumerate}

\end{document}