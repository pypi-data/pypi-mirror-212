\chapter{Troubleshooting}

\subsubsection{The computer crashed in the middle of an experimental session}
\texttt{pychoacoustics} saves the results at the end of each block, therefore  only the results from the last uncompleted block will be lost, the results of completed blocks will not be lost. If you have an experiment with many different blocks presented in random order it may be difficult to see which blocks the listener had already completed and set \texttt{pychoacoustics} to run only the blocks that were not run. To address this issue \texttt{pychoacoustics} keeps a copy of the parameters, including the block presentation order after shuffling, in a file called \texttt{.tmp\_prm.prm} (this is a hidden file on Linux systems). Therefore, after the crash you can simply load this parameters file and move to the block position that the listener was running when the computer crashed to resume the experiment. 

A second function of the \texttt{.tmp\_prm.prm} file is to keep a copy of parameters that were stored in memory, but not saved to a file. If your computer crashed while you were setting up a parameters for an experiment that were not yet saved (or were only partially saved) to a file, you can retrieve them after the crash by loading the \texttt{.tmp\_prm.prm} file. One important thing to keep in mind is that the \texttt{.tmp\_prm.prm} will be overwritten as soon as new parameters are stored in memory by a \texttt{pychoacoustics} instance opened in the same directory. Therefore it is advisable to make a copy of the \texttt{.tmp\_prm.prm} file renaming it to avoid accidentally loosing its contents after the crash.



%%% Local Variables: 
%%% mode: latex
%%% TeX-master: "pychoacoustics_manual"
%%% End: 